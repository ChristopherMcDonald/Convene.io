\documentclass[11pt, oneside]{article} 
\usepackage{geometry}
\geometry{letterpaper}

\usepackage{graphicx}   
\usepackage{amssymb}

\title{ Product Requirements \\ Convene.io}
\author{Christopher McDonald}
\date{Last compiled on \today}

\begin{document}

\maketitle
\tableofcontents
\section{Product Overview}
\subsection{Purpose and Scope}
This product, Convene.io will provide the means of streamlining processes that are typically found during meetings involving various sizes of attendees. This will include the initial planning, distribution of invites, holding the meeting as well as closing it. Traditionally, meetings are seen as a waste of time and can be even more of a waste if not held properly. In order to make meetings more valuable, they must be as fast as possible without sacrificing quality, entice high engagement and only include work which cannot be automated like creative thinking or work done by subject matter experts. \\
\subsection{Definitions, Terms and Acronyms}
\begin{itemize}
\item SaaS: Software-as-a-Service
\item
\end{itemize}
\subsection{Stakeholder Identification}
The stakeholders of this project would include many different people involved in the planning and execution of meetings. They are as follows: \\
\begin{itemize}
	\item \textbf{Department Managers:} They would be a stakeholder as they are heavily involved in making decisions regarding what software is used within their department. All decisions they make should be in the department's best interest and the success of the department is dependant on how well they manage it. Since they will be adopting and implementing our product, they will be affected by the outcome of this product.  
	\item \textbf{Scribes/Note Takers:} Perhaps the heaviest users of this product would be responsible for writing the notes during the meetings. The software they use must work with them to achieve the highest quality of work or it would run the risk of dropping it in favour of another. This is because they would be answering and giving notes to their department managers, who would critique their work. For this product to be successful, it must aid the Scribes in producing their work. 
\end{itemize}
\section{Market Assessment \& Target Demographics}
From current market research, there exists many competitors within this space providing similar services.  These would include Directorpoint, Board Effect and Azeus Convene which will be looked at in depth. They all focus on delivering a product which accompanies users during meetings, with slight different features. The predominant features include scheduling, file-sharing and cross-platform capabilities. Many include integrations with services like SharePoint and Box. Although many do not disclose their price via their website, offering free demos by sales employees is very common. One site revealed they charge \$30 per user, with slightly lower or higher costs depending on the business and features desired. Although not confirmed if in-house options are available, many of them are SaaS applications, offering hosting and delivery of the application of the internet. 
\section{Use Case Diagram}
\begin{figure}[htbp]
   \centering
   \includegraphics[width=0.8\textwidth]{diagrams/UseCase.png}
   \caption{Use Diagram for Convene.io}
   \label{fig:useCaseDiagram}
\end{figure}
% use case list: make meeting, send invites, respond to invite, add text to white board, close meeting, export file, edit meeting
\section{Requirements}
\subsection{Functional Requirements}
\begin{framed}
	\noindent\textbf{Requirement \#}: 1 \hfill \textbf{Requirement Type}: F \hfill\\\\
	\noindent\textbf{Description}: A User must be able to authenticate \\
	\textbf{Rationale}: \\
	\textbf{Fit Criterion}: \\\\
	\textbf{Originator}: \\\\
	\textbf{Priority}: High \hfill \textbf{Conflicts}: None \hfill\\
	\textbf{Supporting Material}: None\\\\
	\noindent\textbf{History}: Created 
\end{framed}
\subsection{Usability Requirements}
\subsection{Technical Requirements}
\subsection{Environmental Requirements}
\subsection{Support Requirements}
\section{Assumptions}
\section{Constraints}
\section{Dependencies}
\section{High-level Workflow Plans}
\section{Evaluation Plan and Performance}


\end{document}  

Requirement Template from github.com/eric73847/Capstone-Project
\begin{framed}
	\noindent\textbf{Requirement \#}: 1 \hfill \textbf{Requirement Type}: F \hfill\\\\
	\noindent\textbf{Description}: \\
	\textbf{Rationale}: \\
	\textbf{Fit Criterion}: \\\\
	\textbf{Originator}: \\\\
	\textbf{Priority}: High \hfill \textbf{Conflicts}: None \hfill\\
	\textbf{Supporting Material}: None\\\\
	\noindent\textbf{History}: Created 
\end{framed}